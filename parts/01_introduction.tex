\part{Introduction}

\chapter{Introduction}

In recent years, a vast amount of data about various types of molecules became
available. For example, we can obtain the complete human genome of a selected
individual in a few days, and about 150 thousand biomacromolecular structures
have been determined and published (Protein Data Bank \cite{Berman2014}).
Furthermore, more
than 100 million various small molecules are described in freely accessible
databases (e.g., Pubchem [], ZINC [], ChEMBL []). This richness of data caused
the formation of novel modern life-science research fields focused on the
utilization of this data. The best-known modern life sciences are
bioinformatics, structural bioinformatics, systems biology, genomics,
proteomics, and also chemoinformatics. These current research specializations
have provided many key results in basic and applied research (e.g. [6–12]).

One fascinating and beneficial field utilizing and processing newly available
data about small molecules (i.e., drug-like compounds) is chemoinformatics.
This discipline offers methodologies for comparing molecular similarity,
molecular database search, virtual screening, and the prediction of molecules'
properties and activities. This prediction is based on the idea that molecular
structures' similarity has a consequence -- a similarity in molecular
properties. In chemoinformatics, the structure is first described using
mathematical characteristics (so-called descriptors) -- numbers containing 3D
(or 2D or 1D) structure information and applicable as inputs of mathematical
models. Then, these models are constructed based on a relation between
descriptors and known values of the property or the activity. Such models are
called Quantitative Structure-Property Relationship (QSPR) models or
Quantitative Structure-Activity Relationship (QSAR) models.

A property, which is strongly required and is therefore often a target of
chemoinformatics prediction models is the acid dissociation constant, $K_a$,
and its negative logarithm p$K_a$. Those p$K_a$ values are of interest
in chemical, biological, environmental, and pharmaceutical research [58–60].
p$K_a$ values have found applications in many areas, such as evaluating and
optimizing drug candidate molecules, pharmacokinetics, ADME profiling,
understanding protein-ligand interactions, etc. Moreover, the critical
physicochemical properties such as permeability, lipophilicity, solubility,
etc., are p$K_a$ dependent. Unfortunately, experimental p$K_a$ values are
available only for a limited set of molecules. In addition to that, obtaining
experimental p$K_a$ values for newly designed molecules is very time-consuming
because they must be synthesized first. Chemoinformatics approaches for p$K_a$
prediction are therefore currently intensively examined.

For this reason, I also focused on the chemoinformatics way of p$K_a$
prediction in my work. Very promising descriptors for p$K_a$ prediction are
partial atomic charges [] because they hold information about the distribution
of electron density within the molecule. Specifically, electron densities on
atoms close to the dissociating hydrogen provide a clue about its dissociation
ability. The most common and accurate method for calculating partial atomic
charges is an application of quantum mechanics (QM). QM calculation can be
performed via various approaches, introducing different approximation levels
(i.e., approximating a wave function by different sets of mathematical
equations, which are called basis sets). QM outputs electron distribution
in orbitals and this distribution can be divided into individual atoms using
several charge calculation schemes (e.g., MPA, NPA, AIM, Hirshfeld, MK, etc.).
Therefore, the correlation between p$K_a$ and relevant atomic charges
calculated by different QM approaches has been analyzed []. I also focused
on this file in my bachelor thesis, developed a workflow for calculation
of p$K_a$ using QM partial atomic charges and examined, which types of QM are
the most suitable [].

QM charges are accurate, but their calculation is very time-consuming. A faster
Alternative to QM charges is empirical charge calculation approaches.
Furthermore, if we would like to apply chemoinformatics p$K_a$ prediction models
practically -- for example, in pre-screening large sets of drug candidates -- we
need a fast approach. Therefore, in my master thesis, I developed a p$K_a$
prediction workflow based on charges (including Electronegativity Equalization
Method).

However, several pieces of the puzzle were still missing. For example, the
developed p$K_a$ prediction workflows [] were strongly dependent on 3D structure
source, and also, the quality of available EEM charges was low.

Therefore, my dissertation's goal was to develop a workflow that predicts p$K_a$
for molecules not synthesized yet and without available experimental 3D
structures.

Specifically, the thesis examined how to improve the process of p$K_a$
prediction via providing suitable inputs. First, the influence of 3D structure
source on p$K_a$ prediction accuracy was analyzed. Afterward, the work focused
on obtaining high-quality partial atomic charges, which served as descriptors
for p$K_a$ calculation. In the end, the authors also support the development
of methodology and software tools for obtaining these high-quality charges.

The thesis structure is the following: First, an overview of key fields is
provided (Chapter 2), i.e. -- 3D structure and approaches for its prediction,
charge calculation methods, and p$K_a$ prediction approaches. Next, the
achieved results, which we published in three research papers, are briefly
described (Chapter 3), and full-texts of the respective published papers are
attached in Part ??. During the elaboration of this thesis, I was also involved
in other projects. Most of them were not related to p$K_a$ prediction but
tightly connected to the field of chemoinformatics or structural bioinformatics.
The outcome of these projects consists of several papers and a book I have
co-authored. Their title pages are attached in Part ??.


