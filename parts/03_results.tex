\part{Results}
\label{part:results}

\chapter{Synopsis of the Results}

We published a series of articles about pKa prediction [] where we showed that
some specific atomic charges correlate with p$K_a$. Based on this, we were able
to build QSPR models for the prediction of this property. We also compared QM,
EEM charges, and their models.  In this dissertation, I focused only
on the last one:

\underline{Geidl S}, Svobodová Vařeková R, Bendová V, Petrusek L, Ionescu C-M,
Jurka Z, Abagyan R, Koča J: \textbf{How Does the Methodology of 3D Structure
Preparation Influence the Quality of p$K_a$ Prediction?} \textit{J Chem Inf Model}
2015, \textbf{55}:1088–1097.

In this article, we utilized different approaches to generate the 3D structures
of organic molecules. These structures were used for the building of p$K_a$
prediction models based on charge descriptors. Then we analyzed various
influences and relationships and found which methodologies for 3D structure
preparation are applicable for p$K_a$ prediction. 

We examined not only pKa prediction models employing QM charges but also the
models utilizing EEM charges. An application of EEM charges looked very
promising. Moreover, EEM charge calculation is significantly faster than
QM charge calculation. 

However, in parallel, we found one significant limitation of EEM charges -- the
parameters. It was available a reasonable number of parameter sets, but they
had only a low coverage.  For example, Bultinc's parameter
set \cite{Bultinck2002} contains parameters only for these elements: C, F, H, N,
O. This fact markedly reduced a dataset, which we were able to use for QM
charges. 

A lack of parameters disallows the usage of EEM charges in many chemoinformatics
applications. For this reason, in our follow-up work, we focused
on the development of new and more robust EEM parameters. The first step was to
develop a new parameterization tool that was easy to use and extendable.
After the prototype, we carefully prepared a new dataset of molecules, and
for this dataset, we computed EEM parameters with higher coverage. These new
parameters were published in a scientific paper:

\underline{Geidl S}, Bouchal T, Raček T, Svobodová Vařeková R, Hejret V,
Křenek A, Abagyan R, Koča J: \textbf{High-quality and universal empirical atomic
charges for chemoinformatics applications.} \textit{J Cheminform} 2015,
\textbf{7}:59.

After some tuning up and extensive research, we released and also published
the tool for EEM parametrization -- the NEEMP software:

Raček T, Pazúriková J, Svobodová Vařekova R, \underline{Geidl S}, Křenek A,
Falginella FL, Horský V, Hejret V, Koča J: \textbf{NEEMP: Software
for validation, accurate calculation and fast parameterization of EEM charges.}
\textit{J Cheminform} 2016.

%TODO: inclue issue and page for last reference!

Sections~\ref{sec:pka-article}, \ref{sec:param-article}, and
\ref{sec:neemp-article} contain a summary and \nameref{chapter:papers} full text of
all these aforementioned articles. I was also involved in other projects, and
the outcome of it is a list of additional articles and book chapters
in \nameref{chapter:publications}.

\section{How Does the Methodology of 3D Structure Preparation Influence
the Quality of pKa Prediction?} \label{sec:pka-article}

From our previous articles [], we know that
the prediction of p$K_a$ is possible via QSPR models using partial atomic
charges as descriptors. The article's goal was to discover how the methodology
of 3D structure preparation influences the quality of pKa prediction.
We prepared a dataset containing 60 phenols, 82 carboxylic acids, 48 anilines,
and additional testing 53 phenols for these purposes. We took structures
from 5 different sources for all these molecules: PubChem, DTP NCI database:
Balloon, Frog2, OpenBabel, and RDKit software. We used neutral forms of all
the molecules and dissociated forms of phenols and carboxylic acids, and
associated forms of anilines. We also optimized these structures with MM
(Molecular mechanics) and QM. All combination led to 7220 structures for that
we calculated four different QM, one semiempirical QM, four different EEM
charges, and Gasteiger-Marselli charges. We created 516 QSPR models for all
these molecules and charges. The robustness of these models was tested by
cross-validation and QM charges also by an independent test set of phenols.

We confirmed that QM and EEM charge descriptors could be used for p$K_a$
prediction. About half of all models have excellent quality with $R^2 >= 0.9$.
We also showed that ab initio and semiempirical charges correlate with p$K_a$
and their models are very accurate. In EEM, we had models with a little worse
quality, but empirical charges are calculated much faster, and an application
in chemoinformatics is much more appropriate. In our models, we were not able
to use Gasteiger-Marsili charges to get an adequate quality. 

We focused on different types of influence. For classes of the benzene
derivates (phenols and anilines) was much easier to obtain high-quality models.
Nevertheless, for aliphatic hydrocarbon derivates (carboxylic acid), it was more
challenging.

The focus of this article was a comparison of the source of the 3D structure.
The influence of input structure for models is essential because
the result -- the quality of QSPR models -- depends on input structures and
their quality. For example, structures taken from RDKIT generated only by
distance geometry produced fragile models. On the other hand, the 3D structures
from the DTP NCI and PubChem databases, formally structures generated by CORINA
and Omega, exhibited the best performance for all the tested molecular classes
and charge calculation approaches. Structures generated by Frog2 also performed
very well. Other 3D structure sources can also be used, but with caution.

We also tried the influence of structure optimization on the quality of QSPR
models. In most cases, differences between original structures and optimized
structures were slight. 

In this article, we summed up the best workflow for the fast and accurate
prediction of pKa. This or similar workflow can also be easily applied to other
important properties for in silico designed molecules. Flow is about preparing
3D structures by CORINA or Omega (with no further optimization), calculation of
EEM charges for these structures, and then the EEM QSPR calculation of p$K_a$.

\subsection{My contribution}

I prepared the input dataset (by extension of published datasets), performed
all the calculations, participated in the analysis of the results, and wrote
a part of the manuscript, including all tables and graphics.

\section{High-quality and universal empirical atomic charges for
chemoinformatics applications} \label{sec:param-article}

The EEM method for charge calculation was published several decades ago.
Before our study, there were done some improvements of EEM [], parameterizations
of empirical parameters [], and developments of new parameterization methods [].
However, there was a problem with the usability of EEM because the available
parameter sets had a low coverage in chemical space.    

We prepared a set of 4475 distinct small organic, drug-like molecules containing
these elements: H, C, N, O, F, P, S, Cl, Br, and I, in different functional
groups. This set was created so that each selected atom type is contained
in at least 100 molecules. CORINA calculated the 3D structure for all molecules.
The next step was the calculation of reference QM charges. We selected
6 different approaches: B3LYP/6-311G/MPA, B3LYP/6-311G/NPA, B3LYP/6-311G/AIM,
HF/6-311G/MPA, HF/6-311G/NPA, and HF/6-311G/AIM. EEM parameterization was
performed for six QM charge calculation approaches, and the whole
set of prepared molecules was used.

Our new EEM parameters get very high quality -- all Pearson coefficients had
a value greater or equal to 0.9. We also showed that the used QM approach did
not prove any difference in the quality of parameters -- B3LYP and HF produced
comparable results. EEM parameters based on NPA and AIM population analysis are
slightly better than EEM parameters based on MPA.

 We also calculated coverage of our parameters previously published parameters
 on four big chemoinformatics databases of drug-like molecules — DrugBank,
 ChEMBL, PubChem, and ZINC. We found out that their coverage is less than 60\%
 for most of the previously published parameters.  Our newly produced parameters
 showed coverage of at least 90\% for these databases. Consistency of coverage
 points out that this problem is not related to a database but concerns
 a chemical space of drug-like molecules and their atom types.

For evaluation of quality, we selected 657 approved drugs from the DrugBank
database. We compared the coefficient of determination, root mean square error
and found out that our new parameters outperform the previously published
parameters. Coverage of the old parameters on this small evaluation dataset
is like coverage on whole databases.

The quality of EEM parameters is also affected by a used QM charge scheme.
EEM parameters derived from MPA, NPA, and AIM charges showed high quality.
EEM parameters based on Hirshfeld charges were acceptable, and MK and CHELPG
charges cannot be used with EEM.  On the other hand, none of the QM theory level
and basis set combinations showed any problem in the quality of EEM parameters.
This also confirmed that we used an appropriate selection of reference QM
charges.

We also evaluated already existing tools for calculating partial atomic charges,
and all the tools showed some issues. For example, OpenBabel tool is using
Bult2002\_mpa parameter set \cite{Bultinck2002}, but developers extended this
set about missing atom types with parameters for different atom types. This hack
increases coverage paid by decreased EEM charges quality for molecules
containing atom type missing in the original parameter set.  

\subsection{My contribution}

I participated in the study's design, and I cooperated in the preparation of
the input data (molecules and published EEM parameters) and in QM charge 
calculation. I performed the analyses and the interpretation of the data.

\section{NEEMP: software for validation, accurate calculation, and fast
parameterization of EEM charges} \label{sec:neemp-article}

This article describes NEEMP -- a software tool with three main
functionalities -- parametrization of EEM charges from reference QM charges,
validation of EEM parameter sets (including quality and coverage calculation),
and EEM charge calculation.

NEEMP provides two different parametrization approaches:

\begin{itemize}
    \item linear regression (LE),
    \item differential evolution with the local minimization (DE-MIN) approach. 
\end{itemize}

A combination of a global optimization method with a local optimization method
improves EEM parameterization. This combined approach provides a more robust
methodology than LR. Therefore it is applicable even for heterogeneous training
sets. Specifically, we combined differential evolution (DE) with the local
minimization method NEWUOA [47]. Quality criteria for evaluation of each 
iteration of the parametrization process (both LR and DE-MIN) can be set up to
coefficient of determination ($R^2$) or the root mean square error (RMSE).

The validation mode of NEEMP calculates quality metrics, coverage, and a
graphical representation of EEM charge correlation with reference QM charges.

The calculation mode of NEEMP provides a calculation of EEM charges using an
input parameter set.

The article also presents two case studies to show the functionality and
performance of NEEMP – a parametrization and a validation case study.

The parametrization case study targets a comparison of the parameterization
method (LR vs. DE-MIN) and metrics for model validation ($R^2$ vs. RMSE).
The case study proved that LR (with both metrics) is suitable for smaller
and homogeneous datasets. DE-MIN (with RMSD metric) is a more robust approach
that can also handle the parametrization of larger and more heterogeneous
datasets. The validation case study provided similar findings to the previous
article –low coverage of the older parameter sets. Also, a quality validation
agrees with the previous article for smaller datasets with molecules comprised
of C, H, N, and O. On the other hand, the case study uncovered an interesting
problem: in larger and more heterogeneous datasets -- the parameters set from
our previous article proved accuracy problems with some atom types. Using NEEMP,
we computed parameter sets, which are also applicable for such problematic
datasets.

\subsection{My contribution}

I prototyped the DE-Min approach and designed some new NEEMP features, such as
the usage of RMSE instead of $R^2$. I also implemented a preparation of validion
reports.

\chapter{Follow-up work and future plans}

From these results, we were able to create a tool for the calculation
of charge descriptors -- ChrgDescCalc.py [Hejret2016], and we also
successfully prepared the universal model for p$K_a$ prediction [Hejret2017].

The EEM parameters computed in [High-quality and universal ...] and [NEEMP]
became a part of AtomicChargeCalculator [] and also its successor 
AtomicChargeCalculator II [].

A parameterization approach from an article [NEEMP], was further extended
by my coworker J. Pazúriková in an article [Pazúriková, J., Křenek, A. and
Matyska, L., 2016. Guided Optimization Method for Fast and Accurate Atomic
Charges Computation]. A further extension of the parameterization approach
(optGM) was done by my colleague T. Raček and it allowed a parameterization
of the majority of empirical charge calculation methods (not only EEM).
An article describing optGM is now in a review process.

In the future, the development of an empirical charge calculation approach
for proteins and a parameter set fully covering Protein Data Bank is
in development.

