\part{Conclusion}

\chapter{Conclusion}

The acid dissociation constant (p$K_a$) is an important property of organic
molecules. Its prediction (especially for unsynthesized molecules) is beneficial
in the drug design process and other modern life science fields.

Our previous articles \cite{Svobodova2011, Svobodova2013} (before my
dissertation) proved that p$K_a$ is successfully predictable by
chemoinformatics models based on partial atomic charges.

In my thesis, I focused on the first topic - analysis of 3D structure sources
on p$K_a$ prediction. We proved that the source of the 3D structure had
a significant impact on charges and, respectively, on the quality of p$K_a$
prediction models. The models exhibited the best performance for two databases
and two software used by these databases for 3D structure generation -- 
a database DTP NCI (where CORINA generates 3D structures) and a database
PubChem (3D structures generated by Omega). Other software tools for 3D
structure generation required additional MM optimization to produce acceptable
or good p$K_a$ prediction models. In this work, we also showed that p$K_a$
prediction models had the best performance when QM or EEM charges (with specific
parameter sets) were used. Purely empirical and topological charges (e.g.,
Gasteiger-Marseli) proved as too approximated for pKa prediction. p$K_a$
prediction models based on EEM charges seemed very promising because they were
fast (no time-demanding QM charge calculation was required), and quality was
high (comparable to models based on QM charges). However, we also found that
the applicability of EEM parameters for drug-like molecules (e.g., if the
parameters cover all atomic types present in the molecule) was significantly
limited. 

It motivated us to focus on the development of EEM parameters suitable 
for p$K_a$ prediction. Specifically, we prepared a new molecule dataset and
successfully computed EEM parameter sets with more extensive coverage
for drug-like molecules and excellent quality ($R^2 > 0.9$). These newly published
parameter sets can be easily used in chemoinformatics applications
such as virtual screening or QSAR/QSPR modeling. We also prepared an OpenBabel
patch with these parameter sets.

During EEM parameters preparation, I realized a need to have a tool for partial
charge calculation and parametrization. For this reason, I focused
on the development of NEEMP software. NEEMP is the only available tool that
provides EEM parametrization, validation of EEM parameters, and calculation
of EEM charges. In NEEMP, we also included an improved parametrization process,
including the DE-MIN method that can markedly increase the quality of final
parameters for heterogeneous datasets. We published NEEMP, and in the article,
we also provided two case studies demonstrating NEEMP capabilities. The
publication also included new EEM parameters tailored for ligand molecules.

These articles together provide a solid base for preparing chemoinformatics
workflows for p$K_a$ prediction, including 3D structures and partial atomic
charges. They sum up the current state of the art and distill the
best of well-known approaches, tools, and parameters to increase the quality
of the final result.


