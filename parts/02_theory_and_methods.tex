\part{Theory and Methods}
\label{part:theory}

\chapter{Structure}

\section{Molecular Structure in Computer}

The chemical structure is the essential information for chemoinformatics and
computational chemistry calculations. We recognize different types of chemical
structures according to the complexity of information \cite{Gasteiger2006}. 

The empirical or chemical formula provides information about molecule
composition -- elements and their count. The structural formula (2D structure)
extends this information about topology -- bonds between atoms. The
three-dimensional structure also provides the conformation of a molecule -- the
relative placement of atoms in space. We try to provide conformation with the
lowest energy representing the most probable conformation of molecule
in reality. For some applications, there can even be an assembly of
these 3D structures.

In chemoinformatics, two-dimensional structures are often used, but the
three-di\-men\-sio\-nal structure can often bring new information into the
\textit{in silico} calculations or models. On the other hand, this 3D structure
can be obtained experimentally for a limited number of small molecules. What
with other molecules, including those which were not synthesized yet?

\section{3D Structure Calculation}

We apply more computationally efficient methods for 3D structure computation
because we use them as input for high-throughput methods. For this reason, many
resources were devoted to the development of fast and accurate 3D structure
prediction methods \cite{Sadowski2008}. These can be classified into the following
groups \cite{Sadowski2008}: rule-based and data-based, fragment-based, numerical methods, and
conformational analysis. These rule-based and data-based, fragment-based
methods are partially overlying.

\subsection{Rule-Based and Data-Based Methods}

These methods use chemical knowledge of geometric and energetic rules known
from experiments and theoretical calculations. In these methods, we use rules
explicitly to describe, e.g., bond lengths and angles; we use data implicitly
to describe, e.g., ring conformation.

\subsection{Fragment-Based Method}

The fragment-based method is the incremental method using rules in the first
step to fragment a structure into parts. According to the following rules,
the parts are assembled by linking fragment templates from a library
(database). Predicted structures are created from the most similar and largest
fragments in a database as possible.

\subsection{Numerical Method}

These numerical methods consist of three methods: molecular mechanics
(MM), quantum mechanics, and distance geometry (DG). Distance geometry is
a great tool to prepare a reasonable initial structure, which is very close
to some low-energy conformation. For this structure, we can use
the optimization process from MM or/and QM.

\subsection{Conformational Analysis}

This method generates a set of conformations for one molecule using
different approaches - genetic algorithms, systematic methods, random
techniques, Monte Carlo or MD simulation. The one or more different structures
are selected based on criteria such as the number of conformers, minimum
RMSD, only conformations with the lowest MM energy (low-energy conformers).

\chapter{Partial Atomic Charges}

\section{The Concept of Atomic Charges}

Atomic charges are a theoretical concept for the quantitative description of
electron density around every atom in a molecule. The first basic concept came
from early chemistry, where an integer expressed these charges (e.g. -1, +2).
Later, they were a real number (partial charge) in organic chemistry and
physical chemistry \cite{Atkins2011}. It is a great approach to explain the
mechanism of a lot of chemical reactions. Recently, partial atomic charges also
became popular in chemoinformatics, as they proved to be informative descriptors
for QSAR and QSPR modeling \cite{Chaves2006, Gross2002} and for other
applications \cite{Moller2005, Zhang2006, Ghafourian2000}; they can be utilized
in virtual screening \cite{Galvez1994, Stalke2011} and similarity
searches \cite{Lyne2002, Bissantz2000}. In reality, we are not able to measure
these numbers, only calculate or estimate them. For such reasons, many different
approaches for the calculation of partial atomic charges were developed.

\section{Overview of Charge Calculation Methods}

\subsection{QM Charge Methods}

These methods use a wave function as a starting point and then apply subsequent
population analysis, charge calculation scheme, or fit to some physical
observation \cite{Cramer2005}. 

Mulliken population analysis (MPA) \cite{Mulliken1955, Mulliken1955a} simply
calculates a charge of an atom as a sum of an electron density from its
molecular orbitals and a half of an electron density from its bonding orbitals.
Natural population analysis (NPA) \cite{Reed1983, Reed1985} sophisticatedly
improves the MPA method by orthogonalization of specific atoms and after this,
NPA performs charge assignment from electron density the same way as in MPA.
NPA atomic charges are more stable and independent of the size of basis sets.
Other possible population analyses are Löwdin population
analysis \cite{Lowdin1950}, Hirshfeld population analysis \cite{Hirshfeld1977}.

AIM (atoms-in-molecules) charge calculation scheme is based on the idea that
electron density measured by X-ray can help with the calculation of partial
charges. Bader and his coworkers \cite{Bader1985, Bader1991} defined an atomic
volume that is used for charge calculation. Other well-known approaches are
electrostatic potential fitting methods (ESP) like CHELPG \cite{Breneman1990}
or MK (Merz-Singh-Kollman) \cite{Besler1990} and their extension -- RESP
methods \cite{Bayly1992}.

Cramer and at \cite{Marenich2009} also developed semiempirical methods -- charge
model 5 (CM5), which extends Hirshfeld population analysis by empirical
parameters to reproduce charge-dependent observables. 

\subsection{Empirical Methods}

Empirical approaches use only empirical parameters, and some of these can
calculate charges from the 3D structure or only from the topology (2D structure)
of a molecule. Therefore, they are distinctly faster than QM approaches. 

One of the first empirical methods developed, CHARGE \cite{Abraham2004}, performs a breakdown
of the charge transmission by polar atoms into single-bond, double-bond, and 
triple-bond additive contributions. Other empirical methods have been developed
on the electronegativity equalization principle. One group of these empirical
approaches are using the Laplacian matrix formalism and the product is
a redistribution of electronegativity: Gasteiger-Marsili (PEOE, partial
equalization of orbital electronegativity) \cite{Gasteiger1980},
GDAC (geometry-dependent atomic charge) \cite{Cho2001}, KCM (Kirchhoff charge
model) \cite{Oliferenko2006}, DENR (dynamic electronegativity
relaxation) \cite{Shulga2008} or TSEF (topologically symmetric energy
function) \cite{Shulga2008}.

The second group of approaches applies the full equalization of orbital
electronegativity. For example, this group contains EEM (electronegativity
equalization method) \cite{Mortier1986} and its extensions (
ABEEM \cite{Yang1997}, SFKEEM \cite{Chaves2006}), QEq (charge
equilibration) \cite{Rappe1991}, EQEq (extended QEq) \cite{Wilmer2012}, or
SQE (split charge equilibration) \cite{Nistor2006}.

Group of conformationally independent methods (based on the 2D structure)
contains CHARGE, Gasteiger-Marsili, KCM, DENR, and TSEF. Group of
conformationally dependent -- geometrical charges (based on the 3D structure)
also consider an influence of conformation and includes the following methods:
GDAC, EEM, ABEEM, SFKEEM, QEq, EQeq, and SQE.

A typical representative of the topological method is the Gasteiger-Marsili
method, which first assigns charges based on atom types and then iteratively
updates atomic charges based on the closest partners. The correction is smaller
and smaller in every step until the sixth step when these corrections are too
small and atomic charges are final. Empirical parameters for this method were
calculated from QM. 

On the other hand, the EEM method needs a complete 3D structure and more
applicable charges for some of the applications.

\section{EEM Calculation}

EEM (electronegativity equalization method) \cite{Mortier1986} is one of the
most popular empirical charge calculation methods and was developed more than
twenty years ago. This method's new parameterizations \cite{Mortier1986, Baekelandt1991, 
Bultinck2002, Bultinck2004, Chaves2006, Svobodova2007, Jirouskova2009, Ouyang2009} and
extension \cite{Chaves2006, Yang1997} are still under development. An advantage of EEM
calculation is that it considers the influence of the molecule's conformation
on the atomic charges. For this reason, EEM charges are often used
in predictive models as chemoinformatics regressors (descriptors) \cite{Todoschini2009}.

EEM is based on three principles: 

The first principle is Sanderson's electronegativity equalization principle.
It assumes that the effective electronegativity of each atom in the molecule
is equal to the molecular electronegativity:

\begin{equation}
    \chi_1 = \chi_2 = \cdots = \chi_x = \bar{\chi} 
\end{equation}
where $\chi_x$ is the effective electronegativity of the atom $x$ and
$\bar{\chi}$ is the molecular electronegativity.

The second principle is the principle of the charge balance. The sum of all
charges is equal to the total charge $Q$:

\begin{equation}
    \sum_{x=1} q_x = Q
\end{equation}
where $q_x$ is the charge of the atom $x$.

And the last principle is the principle of charge-dependent electronegativity.
This principle is the definition of atomic electronegativity, and states that
the electronegativity of each atom can be expressed as a function of its charge: 

\begin{equation}
    \chi_i = A_i + B_i \cdot q_i + \kappa \sum^N_{j=1 \: i\neq{j}} \frac{q_j}{R_{i,j}} 
\end{equation}
where $R_{i,j}$ is the distance between atoms $i$ and $j$, and the coefficients
$A_i$, $B_i$ and $\kappa$ are empirical parameters. 

These principles can be summed up to a system of equations with N + 1 unknowns
(where $q_1$, $q_2$, ... , $q_N$ and $\bar{\chi}$):

\begin{equation} \label{eq:eem}
    \left(
    \begin{array}{ccccc}
        B_1                    & \frac{\kappa}{R_{1,2}} & \cdots & \frac{\kappa}{R_{1,N}} & -1     \\
        \frac{\kappa}{R_{2,1}} & B_2                    & \cdots & \frac{\kappa}{R_{2,N}} & -1     \\
        \vdots                 & \vdots                 & \ddots & \vdots                 & \vdots \\
        \frac{\kappa}{R_{N,1}} & \frac{\kappa}{R_{N,2}} & \cdots & B_N                    & -1     \\ 
        1                      & 1                      & 1      & 1                      & 0      \\  
    \end{array}
    \right) \cdot
    \left(
    \begin{array}{c}
        q_1        \\
        q_2        \\
        \vdots     \\
        q_N        \\
        \bar{\chi} \\
    \end{array}
    \right) =
    \left(
    \begin{array}{c}
        -A_1   \\
        -A_2   \\
        \vdots \\
        -A_N   \\
        Q      \\
    \end{array}
    \right)
\end{equation}
The first values of parameters $A_i$ and $B_i$ were modifications
of experimental hardness and electronegativity \cite{Mortier1986}. $\kappa$ was
equal to 1. Nowadays, these parameters are calculated from the QM
charges \cite{Baekelandt1991, Bultinck2002, Bultinck2004,
Chaves2006, Svobodova2007, Jirouskova2009, Ouyang2009}.
Therefore, EEM charges were correlated with the QM
charge calculated with the same method used for parametrization.

\section{Quality and Usability of EEM parameters}

The quality of EEM parameters describes how the empirical charges computed
using these EEM parameters correspond with QM charges used for EEM
parameterization. Three main characteristics (statistics \cite{Urdan2011, Verzani2018})
can describe the quality of EEM parameters -- the coefficient of determination
root mean square error (RMSE) and average absolute error ($\bar{\Delta}$).

\textbf{The coefficient of determination $R^2$} is the squared value of the
Pearson coefficient (equation \ref{eq:coef_det}). This value describes the
linear correlation rate. Values close to 1 mean that values correlate very well,
and values close to 0 mean no correlation.

\begin{equation} \label{eq:coef_det}
    R = \sqrt{\frac{
        \sum^N_{x=1}{((q^{calc}_x - \overline{q}^{calc}_x) \cdot (q^{ref}_x - \overline{q}^{ref}_x))}
    }{
        \sum^N_{x=1}{(q^{calc}_x - \overline{q}^{calc}_x)^2} \cdot \sum^N_{x=1}{(q^{ref}_x - \overline{q}^{ref}_x)^2}
    }}
\end{equation}
where $q^{ref}$ is the reference value of charge calculated by QM and $q^{calc}$
is charge value calculated by EEM. $\overline{q}^{ref}$, $\overline{q}^{calc}$
are the average value of $q^{ref}$, respectively $q^{calc}$.

\textbf{Root mean square error} RMSE is the normalized sum of squared error
describing the reliability of the model calculated by:

\begin{equation}
    \mathrm{RMSE} = \frac{
        \sum^N_{x=1}{(q^{calc}_x - q^{ref}_x)^2}
    }{
        N
    }
\end{equation}

\textbf{Average absolute error} $\overline{\Delta}$ is an averaged difference
between corresponding EEM and QM charges in a molecule and is calculated
according to an equation:

\begin{equation}
    \overline{\Delta} = \frac{
        \sum^N_{x=1}{|q^{calc}_x - q^{ref}_x|}
    }{
        N
    }
\end{equation}

Their \textbf{coverage} describes the applicability of EEM parameters.
Coverage is a percentage value describing EEM parameters' ability to calculate
charges for individual molecules in a dedicated dataset. \textit{De facto},
this coverage depends on the representation of atom types in EEM parameters.

\begin{equation}
    \mathrm{coverage} = \frac{N_{pos}}{N_{tot}}
\end{equation}
where $N_{pos}$ is the number of molecules able calculated by EEM parameters
and $N_{tot}$ is the total number of molecules in a dataset.

\section{EEM Parametrization}

For the parameterization of EEM charges, a lot of different methods have been
introduced. We can summarize it into two groups: one group contains
a method that analytically solves equation \cite{eq:eem} -- linear
regression \cite{Svobodova2007, Jirouskova2009} and the
second group contains optimization methods \cite{Ouyang2009}  such as
Accelerated Random Search, Particle Swarm Optimization, and Differential
Evolution algorithms. Both of these groups need input -- a set of molecules
with 3D structures and QM atomic charges. In my work, linear regression and
differential evolution were used, and therefore, they are described in more
detail below:

\textbf{The linear regression (LR)} method is based on these two equations:

\begin{equation} \label{eq:eem1}
    A_i + B_i \cdot x = y
\end{equation}

\begin{equation} \label{eq:eem2}
    \begin{array}{cc}
    x =& q_i \\
    \vspace{3mm}
    y =& \chi_i - \kappa \sum^N_{j=1 \: i\neq{j}} \frac{q_j}{R_{i,j}} \\
    \end{array}
\end{equation}

Equations are derived from equations \ref{eq:eem1} and \ref{eq:eem2}, which
define the EEM method. In the LR method, the dataset of molecules with QM
charges can change in every iteration to improve the quality of resulting
charges. Quality criterium can be the Pearson correlation coefficient or
the coefficient of determination, and the root mean square error or different
types of errors.  An advantage of the LR method is its straightforwardness and
the possibility to optimize $\kappa$ by another iteration. On the other hand,
this method is not possible to make parametrization for some extensions of EEM
like SFKEEM and ABEEM.

\textbf{Differential Evaluation (DE)} \cite{Storn1997} is a heuristics method
to focus on finding a global minimum of a function. This method works similar
to other optimization methods -- iteratively optimize parameters to improve
the final solution. Parameters of function are set up randomly, mutated, and
evaluated until there is no best solution.

\chapter{Acid Dissociation Constant Prediction}

\section{Motivation}

The acid dissociation constant (p$K_a$) is a physicochemical property that
characterizes the strength of acids. It is one of the essential properties
for pharmaceutical, chemical, biological and environmental research or industry.
For example, it can be used in the chemoinformatics pipeline for evaluation and
optimization of drug candidate \cite{Ishihama2002, Babic2007, Manallack2007},
ADME profiling \cite{Wan2006, Cruciani2009}, pharmacokinetics \cite{Comer2001},
understanding protein-ligand interactions \cite{Klebe2000, Lee2009}.

\section{Overview of Approaches}

Several different approaches for pKa prediction have been
developed \cite{Lee2009, Rupp2010, Fraczkiewicz2006, Ho2010}. 

\subsection{LFER (Linear Free Energy Relationships) Methods}

This is one of the oldest methods \cite{Clark1964, Perrin1981} for p$K_a$
prediction. This method uses the linear relation of Gibbs energy and p$K_a$ or
the logarithm of a reaction rate constant -- the Hammett and Taft equations.
An advantage of this method is a simple, straightforward, and quick calculation,
but on the other hand, we need substituent and reaction parameters.
This method is still used in the programs ACD/pKa \cite{acd},
EPIK \cite{Shelley2008}, and SPARC \cite{Hillal1995}.

\subsection{Database Methods}

These methods \cite{Sayle, Blower2006} use a library (database) of molecules
with known pKa values. The p$K_a$ value is taken directly from this library, or 
it is interpolated or triangulated from most similar molecules in this library.
Most accurate results are produced only for molecules that are similar to
molecules in the database. For this reason, it is essential to have
an extensive library.

\subsection{Ab Inition Quantum Mechanical Calculations}

These methods \cite{Liptak2002, Toth2001} use the fact that the dissociation 
constant can be calculated from the Gibbs energy of the reaction and from the
solvation based on equation~\ref{eq:ka}. However, there is no general approach,
and every specific calculation configuration needs to be calibrated based on
experimental values. The significant disadvantage of these methods is that they
are time-consuming. On the other hand, these methods can be very accurate if
they use correct calibration parameters. It is only one of the few methods that
can be used to extend the training dataset with experimental values or validate
some of this experimental value. It means that other methods can be improved
by this method. This method is implemented as a module of the Jaguar quantum
chemical software package \cite{jaguar}.

\begin{equation}
    \mathrm{p}K_a = - \log_{10} K_a
\end{equation}

\begin{equation} \label{eq:ka}
    K_a = e^{\frac{-\Delta G^\circ}{RT}}
\end{equation}

\subsection{QSPR Method}

The quantitative structure-property relationship method \cite{Dixon1993,
Zhang2006, Jelfs2007} uses mainly
a linear model to describe the relationship between molecular structure and
a property of a molecule, in our case p$K_a$. In those models, structures are
presented by descriptors \cite{Todoschini2009} that are numerical expressions of molecular
properties. For example, descriptors can be the number of hydrogen atoms, the 
ratio between carbon atoms and all atoms in the molecule, or solvent accessible
surface area.

p$K_a$ correlates well with the polarizability, HOMO energy \cite{Gross2001},
proton-transfer energy \cite{Gross2001},
partial atomic charges \cite{Citra1999, Gross2002, Kreye2009, Svobodova2011, Svobodova2013},
the electrostatic potential of the molecule \cite{Liu2009}, etc.
Partial atomic charges proved as very promising
descriptors \cite{Citra1999, Gross2002, Kreye2009, Svobodova2011, Svobodova2013} for p$K_a$
prediction.


